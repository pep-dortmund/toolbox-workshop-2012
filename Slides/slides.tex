\documentclass[t]{beamer}

\usetheme{TU}

\hypersetup{unicode=true,pdfcreator={},pdfproducer={}}

\usepackage{booktabs}
\usepackage{array}

\title{Toolbox Workshop}
\subtitle{Nützliche Programme für Physikstudenten}
\author[Igor B.\and Kevin D.\and Christian G.\and Peter L.\and Ismo T.]{
       Igor Babuschkin%\thanks{\href{mailto:igor.babuschkin@udo.edu}{igor.babuschkin@udo.edu}}
  \and Kevin Dungs%\thanks{\href{mailto:kevin.dungs@udo.edu}{kevin.dungs@udo.edu}}
  \and Christian Gerhorst%\thanks{\href{mailto:christiangerhorst@gmail.com}{christiangerhorst@gmail.com}}
  \and Peter Lorenz%\thanks{\href{mailto:peter.lorenz@udo.edu}{peter.lorenz@udo.edu}}
  \and Ismo Toijala%\thanks{\href{mailto:ismo.toijala@udo.edu}{ismo.toijala@udo.edu}}
}
\institute[PeP et al. e.V.]{PeP et al. e.V.\thanks{\href{http://pep-dortmund.org}{pep-dortmund.org}}}
\date{September 2012}

\begin{document}
  {
    \setbeamertemplate{footline}{}
    \begin{frame}
      \titlepage
    \end{frame}
  }

  \begin{frame}{Inhalt}
    \tableofcontents
  \end{frame}

  \section{Unix Shell}
    \subsection{Allgemeines}
      \begin{frame}{Dateisystem}
        \begin{itemize}
          \item \texttt{/} trennt Teile eines Pfads (Ordner und Dateien) (Windows: \texttt{\textbackslash})
          \item Das gesamte Dateisystem bildet \emph{einen} Baum, beginnend mit \texttt{/}
          \item Es gibt immer einen aktuellen Ordner (working directory)
          \item Pfade können absolut (beginnend mit \texttt{/}) oder relativ zum aktuellen Ordner angegeben werden
          \item drei spezielle Ordner:
            \begin{itemize}
              \item \texttt{.} der aktuelle Ordner (oder der aktuelle im bisherigen Pfad, \texttt{a/./}~=~\texttt{a/})
              \item \texttt{..} der Oberordner des aktuellen Ordners (\texttt{a/b/../}~=~\texttt{a/})
              \item \texttt{\textasciitilde} das Heimverzeichnis (nur am Anfang eines Pfads)
            \end{itemize}
          \item Dateien, die mit \texttt{.} anfangen sind versteckt (z.B. \texttt{\textasciitilde/.vimrc})
        \end{itemize}
      \end{frame}

      \begin{frame}{Aufbau einer Eingabe}
        \texttt{\$ ls -l --all \textit{directory}\\
                \textit{output}\\
                \$}
        \begin{center}
          \begin{tabular}{>{\tt}l l}
            \toprule
            \$                 & Prompt       \\
            ls                 & Befehl       \\
            -l                 & kurze Option \\
            --all              & lange Option \\
            \textit{directory} & Argument     \\
            \textit{output}    & Ausgabe      \\
            \bottomrule
          \end{tabular}
        \end{center}
      \end{frame}

      \begin{frame}
        \begin{itemize}
          \item \texttt{\$} ist nur ein Beispiel für einen Prompt, häufig wird das aktuelle Verzeichnis und/oder andere Informationen angezeigt
          \item kurze Optionen können zusammengefasst werden (\texttt{ls~-la} = \texttt{ls -l -a} = \texttt{ls -l --all})
          \item die Reihenfolge der Optionen ist egal
          \item meistens werden mehrere Argumente (z.B. Dateien) akzeptiert
        \end{itemize}
      \end{frame}

    \subsection{Befehle}
      \begin{frame}{\texttt{man}}
        \begin{itemize}
          \item \texttt{man} für manual
          \item zeigt die Hilfe für ein Programm
          \item Beispiel: \texttt{man man}
        \end{itemize}
      \end{frame}

      \begin{frame}{\texttt{pwd}, \texttt{cd}}
        \begin{itemize}
          \item \texttt{pwd}: zeigt das aktuelle Verzeichnis
          \item \texttt{cd}: wechselt in das angegebene Verzeichnis
          \item Beispiel:\\
            \texttt{\$ pwd\\
                    /home/ismo\\
                    \$ cd ../../etc\\
                    \$ pwd\\
                    /etc\\
                    \$ cd \textasciitilde\\
                    \$ pwd
                    /home/ismo}
        \end{itemize}
      \end{frame}

      \begin{frame}{\texttt{mkdir}}
      \end{frame}

      \begin{frame}{ls}
        \begin{itemize}
          \item ls -l\\
          \item ls -a\\
          \item ls -R
        \end{itemize}
      \end{frame}

      \begin{frame}{cp, mv, rm}
        \begin{itemize}
          \item cp -r\\
          \item rm -r\\
          \item rm -f
        \end{itemize}
      \end{frame}

      \begin{frame}{cat, less}
      \end{frame}

      \begin{frame}{grep}
      \end{frame}

      \begin{frame}{Allgemeines}
        \begin{itemize}
          \item pipes\\
          \item ctrl-c\\
          \item ctrl-d\\
          \item .., .\\
          \item >, >>, <\\
          \item glob (*, ...)
        \end{itemize}
      \end{frame}

  \section{git}
    \subsection{Allgemeines}
      \begin{frame}{warum}
      \end{frame}

      \begin{frame}{hoster}
      \end{frame}

    \subsection{Befehle}
      \begin{frame}{init, clone}
      \end{frame}

      \begin{frame}{status, log}
      \end{frame}

      \begin{frame}{add, mv, rm}
      \end{frame}
      
      \begin{frame}{commit}
      \end{frame}
      
      \begin{frame}{push, pull}
      \end{frame}
      
      \begin{frame}{mergetool}
      \end{frame}

  \section{Python}
    \subsection{Sprache}
      \begin{frame}{ipython}
      \end{frame}
      
      \begin{frame}{variablen, operatoren}
      \end{frame}
      
      \begin{frame}{(), [], \{\}}
      \end{frame}
      
      \begin{frame}{if, elif, else}
      \end{frame}
      
      \begin{frame}{for, while}
      \end{frame}
      
      \begin{frame}{def, Funktionsaufrufe}
      \end{frame}
      
      \begin{frame}{import, from, as}
      \end{frame}
    
    \subsection{Bibliotheken}
      \begin{frame}{numpy}
      \end{frame}
      
      \begin{frame}{scipy}
      \end{frame}
      
      \begin{frame}{matplotlib}
      \end{frame}
      
      \begin{frame}{uncertainties}
      \end{frame}
      
      \begin{frame}{pylab}
      \end{frame}
\end{document}
