
\subsection{Bibliotheken}

\begin{frame}{Beispiele und Aufgaben herunterladen}
  Lade das Repository \\
  \url{https://github.com/ibab/toolbox} \\
  per \texttt{git clone} herunter.
\end{frame}

\subsubsection{PyLab}
\begin{frame}{PyLab}
  \begin{itemize}
    \item bündelt NumPy, SciPy und Matplotlib
    \item starten mit \texttt{ipython3 --pylab}
  \end{itemize}
\end{frame}

\subsubsection{NumPy}
\begin{frame}{NumPy}
  \begin{center}
    \includegraphics[width=100px]{../Notes/img/numpy.png}
  \end{center}
  \begin{itemize}
    \item $n$-dimensionale Arrays
    \item Funktionen, die auf denen arbeiten
    \item Operatoren wirken elementweise
    \item wird meist mit \texttt{np} abgekürzt
  \end{itemize}

  Konstanten:
  \begin{itemize}
    \item \texttt{pi}
    \item \texttt{e}
  \end{itemize}
\end{frame}

\begin{frame}{Arrays erstellen}
  \begin{itemize}
    \item \texttt{array}: konvertiert irgendwas (Liste, Tupel, …) zu einem Array
    \item \texttt{linspace(start, end, number)}: Array aus \texttt{number} Zahlen zwischen \texttt{start} und \texttt{end} in gleichem Abstand
    \item \texttt{arange(start, end, step)}: Array aus Zahlen zwischen \texttt{start} und \texttt{end} mit dem Abstand \texttt{step}
    \item \texttt{zeros(shape)}: Array aus Nullen der Größe \texttt{shape}
    \item \texttt{ones(shape)}: Array aus Einsen der Größe \texttt{shape}
  \end{itemize}
\end{frame}

\begin{frame}{Elementweise Funktionen}
  Beispiele:
  \begin{itemize}
    \item \texttt{sqrt}
    \item \texttt{exp}, \texttt{log}
    \item \texttt{sin}
    \item \texttt{deg2rad}, \texttt{rad2deg}
  \end{itemize}
\end{frame}

\begin{frame}{Reduzierende Funktionen}
  Beispiele:
  \begin{itemize}
    \item \texttt{sum}
    \item \texttt{mean}
    \item \texttt{max}, \texttt{min}
    \item \texttt{ediff1d}
  \end{itemize}
\end{frame}

\begin{frame}{It/Output}
  \begin{itemize}
    \item \texttt{loadtxt(file [, unpack=True])}: Lädt eine Datei in ein Array.
      \texttt{unpack=True} transponiert das Array
    \item \texttt{savetxt(file, array)}: Speichert ein Array in eine Datei
  \end{itemize}
\end{frame}

\subsubsection{SciPy}
\begin{frame}{SciPy}
  \begin{center}
    \includegraphics{../Notes/img/scipy.pdf}
  \end{center}
\end{frame}

\begin{frame}{Nützliche Funktionen}
  \begin{itemize}
    \item \texttt{optimize.curve\_fit}: fittet nichtlineare Funktionen
    \item \texttt{stats.sem}: gibt den Fehler des Mittelwerts
    \item \texttt{constants.C2K}: konvertiert Celsius in Kelvin
    \item \texttt{constants.K2C}: konvertiert Kelvin in Celsius
  \end{itemize}
\end{frame}

\begin{frame}{Konstanten}
  \begin{itemize}
    \item \texttt{constants.physical\_constants}: enthält diverse physikalische Konstanten, ihre Fehler und Einheiten (aus CODATA)
  \end{itemize}
\end{frame}

\subsubsection{matplotlib}
\begin{frame}{matplotlib}
  \begin{center}
    \includegraphics[width=\textwidth]{../Notes/img/matplotlib.pdf}
  \end{center}
  \begin{itemize}
    \item prozedurales Interface (pyplot, in PyLab) \mdash\ einfacher
    \item objektorientiertes Interface (Beschreibung im Skript) \mdash\ flexibler, schöner für größere Skripte oder Programme
  \end{itemize}
\end{frame}

\begin{frame}{Beispiele}
  \begin{center}
    \includegraphics[width=0.8\textwidth]{img/matplotlib/hist.pdf}
  \end{center}
\end{frame}

\begin{frame}{Beispiele}
  \begin{center}
    \includegraphics[width=0.8\textwidth]{img/matplotlib/errorbars.pdf}
  \end{center}
\end{frame}

\begin{frame}{Beispiele}
  \begin{center}
    \includegraphics[width=0.8\textwidth]{img/matplotlib/finance.pdf}
  \end{center}
\end{frame}

\begin{frame}{Beispiele}
  \begin{center}
    \includegraphics[width=0.8\textwidth]{img/matplotlib/polar.pdf}
  \end{center}
\end{frame}

\begin{frame}{Beispiele}
  \begin{center}
    \includegraphics[width=0.8\textwidth]{img/matplotlib/tex.pdf}
  \end{center}
\end{frame}

\begin{frame}{Beispiele}
  \begin{center}
    \includegraphics[width=0.8\textwidth]{img/matplotlib/mplot3d.pdf}
  \end{center}
\end{frame}

\begin{frame}[fragile]{Plotten mit IPython}
  Einfach \texttt{ipython3 --pylab} ausführen und z.B. Folgendes eintippen:
\begin{minted}{python}
  In [1]: x = linspace(0, 1, 100)
  In [2]: plot(x, x**2, 'b-')
\end{minted}
  An dem Plot kann interaktiv weiter gearbeitet werden.
\end{frame}

\begin{frame}[fragile]{Plots in .py-Dateien erstellen}
  Erst einmal Bibliotheken importieren.\\
  Plot erscheint, wenn man \texttt{show()} aufruft.
\begin{minted}{python}
from numpy import linspace, pi, sin
import matplotlib.pyplot as plt
x = linspace(0, 2 * pi, 1000)
plt.plot(x, sin(x), 'r--')
plt.show()
# oder: plt.savefig('plot.pdf')
\end{minted}
\end{frame}

\begin{frame}{Verschiedene Arten von Plots}
  \begin{itemize}
    \item \texttt{plot}
    \item \texttt{errorbar} - Plot mit Fehlerbalken
    \item \texttt{semilogy}, \texttt{semilogx} - logarithmische Skalierung
    \item \texttt{hist} - Histogramme
    \item \texttt{polar} - polare Plots
  \end{itemize}
\end{frame}

\begin{frame}{Nützliche Funktionen}
  \begin{itemize}
    \item \texttt{title('...')}
    \item \texttt{xlabel('...'), ylabel('...')}
    \item \texttt{grid()}
    \item \texttt{xlim(a, b)}, \texttt{ylim(a, b)}
    \item \texttt{legend()} (beim Plot \texttt{label='...'} einstellen
    \item \texttt{clf()}
  \end{itemize}
\end{frame}

\begin{frame}{Matplotlib einstellen}
  2 Möglichkeiten:
  \begin{itemize}
    \item direkt in der Code-Datei
    \item Datei \texttt{matplotlibrc} im selben Ordner
  \end{itemize}
  $\Rightarrow$ siehe Dokumentation
\end{frame}
