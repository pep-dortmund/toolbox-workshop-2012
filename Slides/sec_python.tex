\section{Python}
\begin{frame}{Python}
  \begin{center}
    \includegraphics[width=140px]{img/python.png} \\
    \color{TUgreen}\textbf{\href{http://python.org}{www.python.org}}
  \end{center}
  \begin{quote}
    \begin{spacing}{1.0}
      Python is a programming language that lets you work more quickly and integrate your systems more effectively. You can learn to use Python and see almost immediate gains in productivity and lower maintenance costs.
    \end{spacing}
  \end{quote}
\end{frame}

\begin{frame}{Python ist...}
  \begin{itemize}
    \item […] eine Programmiersprache.
    \item […] einfach!
    \item […] sehr mächtig.
    \item […] universell einsetzbar.
  \end{itemize}
  \begin{center}
    \includegraphics[width=120px]{img/eierlegendewollmilchsau.jpg}\\
    \tiny\texttt{\href{http://xn--sptzlemitsoss-cfb.de/wp-content/uploads/2012/07/eierlegendewollmilchsau.jpg}{http://xn--sptzlemitsoss-cfb.de/wp-content/uploads/2012/07/eierlegendewollmilchsau.jpg}}\normalsize
  \end{center}
\end{frame}

\begin{frame}[fragile]{Ein kleines Beispiel}
  \vspace{-1em}
  \begin{columns}
    \begin{column}{0.5\textwidth}
      \begin{exampleblock}{C++}
        \begin{minted}{c++}
#include <iostream>
using namespace std;
int main(int argc, char *argv[])
{
    cout << "Hello, World!" << endl;
    return 0;
}
        \end{minted}
      \end{exampleblock}
    \end{column}
    \begin{column}{0.5\textwidth}
      \begin{exampleblock}{Python}
        \begin{minted}{python}
print("Hello, World!")
        \end{minted}
      \end{exampleblock}
    \end{column}
  \end{columns}
\end{frame}

\begin{frame}{Python}
  \tableofcontents[sectionstyle=show/hide,
                   subsectionstyle=show/show/hide,
                   subsubsectionstyle=show/show/show]
\end{frame}

\subsection{IPython}
\begin{frame}{IPython}
  \begin{center}
    \includegraphics[width=300px]{img/ipython.png}
  \end{center}
\end{frame}

\subsection{Syntax}
\begin{frame}{Syntax}
  \begin{block}{Blöcke}
    \begin{itemize}
      \item Durch Einrückung!
      \item 4 Leerzeichen / 1 Tab
    \end{itemize}
  \end{block}
  \begin{block}{Semikolons}
  \begin{itemize}
    \item Gibt es prinzipiell
    \item Sind am Zeilenende aber nicht notwendig
  \end{itemize}
  \end{block}
\end{frame}

\begin{frame}[fragile]{Variablen}
  \begin{itemize}
    \item Dynamische Typisierung
    \item Keine explizite Deklaration 
  \end{itemize}
  \begin{spacing}{1.0}
    \begin{exampleblock}{Beispiel}
      \begin{minted}{python}
In [1]: a = 1

In [2]: b = 2

In [3]: name = "Kääähbiiin"

In [4]: a, b, name
Out[4]: (1, 2, 'Kääähbiiin')
      \end{minted}
    \end{exampleblock}
  \end{spacing}  
\end{frame}

\begin{frame}{Datenstrukturen}
  \begin{itemize}
    \item bool (\texttt{True}, \texttt{False})
    \item int, float, long, complex
    \item string (\texttt{'foo'}, \texttt{"bar"})
    \item Iteratoren, Generatoren, Sequenzen…
  \end{itemize}
\end{frame}

\begin{frame}[fragile]{Datenstrukturen}
  \begin{block}{Praktische Typen}
    \begin{itemize}
      \item[\texttt{()}] Tupel
      \item[\texttt{[]}] Liste
      \item[\texttt{\{\}}] Dictionary 
    \end{itemize}
  \end{block}
  \vspace{.5em}
  \begin{spacing}{1.0}
    \begin{exampleblock}{Zum Beispiel}
      \begin{minted}{python}
In [9]: cities = ['Dortmund', 'Hamburg', 'Berlin']

In [10]: cities[0]
Out[10]: 'Dortmund'
      \end{minted}
    \end{exampleblock}
  \end{spacing}
\end{frame}

\begin{frame}[fragile]{Mehr Beispiele}
  \begin{minted}{python}
In [1]: teams = {
   ...:         'BVB': "BV Borussia Dortmund 09",
   ...:         'S04': "FC Schalke 04",
   ...:         'FCB': "FC Bayern München"
   ...: }

In [2]: teams['BVB']
Out[2]: 'BV Borussia Dortmund 09'
  \end{minted}
\end{frame}

\begin{frame}[fragile]{Operatoren}
  \begin{spacing}{1.0}
    \begin{block}{Arithmetische Operatoren}
      \begin{minted}{python}
  +, -, *, /, %, **, //
      \end{minted}
    \end{block}
    \begin{block}{Zuweisungsoperatoren}
      \begin{minted}{python}
  =, +=, -=, *=, /=, %=, **=, //=
      \end{minted}
    \end{block}
    \begin{block}{Vergleichsoperatoren}
      \begin{minted}{python}
  ==, !=, <, <=, >, >=
      \end{minted}
    \end{block}
  \end{spacing}
\end{frame}

\begin{frame}[fragile]{Operatoren}
  \begin{spacing}{1.0}
    \begin{block}{Logische Operatoren}
      \begin{minted}{python}
  and, or, not
      \end{minted}
    \end{block}
    \begin{block}{Identitätsoperatoren}
      \begin{minted}{python}
  is, is not
      \end{minted}
    \end{block}
    \begin{block}{Operatoren für Sequenzen}
      \begin{minted}{python}
  in, not in
      \end{minted}
    \end{block}
  \end{spacing}
\end{frame}

\begin{frame}[fragile]{Kontrollstrukturen}
  \begin{spacing}{1.0}
    \begin{block}{if}
      \begin{minted}{python}
if condition:
    # do something
elif other_condition:
    # or do something else
else:
    # or something else
      \end{minted}
    \end{block}
    \begin{block}{case}
    gibt es \emph{nicht}!
    \end{block}
  \end{spacing}
\end{frame}

\begin{frame}[fragile]{Schleifen}
  \begin{block}{while}
    \begin{minted}{python}
while condition:
    # do something
    \end{minted}
  \end{block}
  \begin{block}{for}
  \end{block}
\end{frame}

\begin{frame}{Funktionen}
\begin{block}{Aufrufe}
\end{block}
\begin{block}{def}
\end{block}
\end{frame}

\begin{frame}{Module}
import, from, as
\end{frame}

\begin{frame}[fragile]{Und ohne IPython?}
\begin{itemize}
  \item Dateiendung .py
  \item Aufruf per \texttt{python dateiname.py}
  \item Oder \texttt{chmod +x dateiname.py \&\& ./dateiname.py}
  \item \texttt{print()}-Funktion 
\end{itemize}
  \begin{spacing}{1.0}
    \begin{exampleblock}{Ein besseres "Hello, World!"}
      \begin{minted}{python}
#!/usr/bin/env python
# coding=utf-8

print("Hello, World!")
      \end{minted}
    \end{exampleblock}
  \end{spacing}
\end{frame}

\subsection{Bibliotheken}
\subsubsection{numpy}
\begin{frame}{numpy}
\end{frame}

\subsubsection{SciPy}
\begin{frame}{scipy}
\end{frame}

\subsubsection{matplotlib}
\begin{frame}{matplotlib}
\end{frame}

\subsubsection{PyLab}
\begin{frame}{PyLab}
\end{frame}
