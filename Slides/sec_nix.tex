\section{Unix Shell}
  \begin{frame}{Unix Shell}
    \tableofcontents[sectionstyle=show/hide,
                     hideothersubsections]
  \end{frame}

  \subsection{Dateisystem}
    \begin{frame}{Dateisystem}
      \begin{itemize}
        \item \texttt{/} trennt Teile eines Pfads
        \item Das gesamte Dateisystem bildet \emph{einen} Baum
        \item Es gibt immer ein aktuelles Verzeichnis
        \item Pfade können absolut oder relativ angegeben werden
        \item drei spezielle Verzeichnisse:
          \begin{itemize}
            \item \texttt{.} das aktuelle Verzeichnis
            \item \texttt{..} das Oberverzeichnis des aktuellen Verzeichnisses
            \item \texttt{\textasciitilde} das Heimverzeichnis
          \end{itemize}
        \item Dateien, die mit \texttt{.} anfangen sind versteckt
      \end{itemize}
    \end{frame}

  \subsection{Aufbau einer Eingabe}
    \begin{frame}{Aufbau einer Eingabe}
      \begin{singlespace}
        \texttt{\$ ls -l --all \textit{directory}\\
                \textit{output}\\
                \$}
      \end{singlespace}
      \begin{center}
        \begin{tabular}{>{\tt}l l}
          \toprule
          \$                 & Prompt       \\
          ls                 & Befehl       \\
          -l                 & kurze Option \\
          --all              & lange Option \\
          \textit{directory} & Argument     \\
          \textit{output}    & Ausgabe      \\
          \bottomrule
        \end{tabular}
      \end{center}
    \end{frame}

    \begin{frame}
      \begin{itemize}
        \item \texttt{\$} ist nur ein Beispiel für einen Prompt
        \item kurze Optionen können zusammengefasst werden
        \item die Reihenfolge der Optionen ist egal
        \item meistens werden mehrere Argumente akzeptiert
      \end{itemize}
    \end{frame}

  \subsection{Befehle}
  \begin{frame}{\texttt{man}, \texttt{pwd}, \texttt{cd}, \texttt{echo}}
      \begin{itemize}
        \item \texttt{man \textit{topic}} für manual: zeigt die Hilfe für ein Programm
        \item \texttt{pwd} für print working directory: zeigt das aktuelle Verzeichnis
        \item \texttt{cd \textit{directory}} für change directory: wechselt in das angegebene Verzeichnis
        \item \texttt{echo \textit{text}}: gibt einen Text aus
      \end{itemize}
    \end{frame}

    \begin{frame}{ls}
      \begin{itemize}
        \item \texttt{ls [\textit{directory}]} für list: zeigt den Inhalt eines Verzeichnisses an
        \item \texttt{ls -l}: zeigt mehr Informationen über Dateien und Verzeichnisse
        \item \texttt{ls -a}: zeigt auch versteckte Dateien
        \item \texttt{ls -R}: zeigt auch den Inhalt von Unterverzeichnissen
      \end{itemize}
    \end{frame}

    \begin{frame}{\texttt{mkdir}, \texttt{touch}}
      \begin{itemize}
        \item \texttt{mkdir \textit{directory}} für make directory: erstellt ein neues Verzeichnis
        \item \texttt{mkdir -p \textit{directory}}: erstellt ein neues Verzeichnis und alle notwendigen Oberverzeichnisse
        \item \texttt{touch \textit{file}}: erstellt eine neue, leere Datei
      \end{itemize}
    \end{frame}

    \begin{frame}{\texttt{cp}, \texttt{mv}, \texttt{rm}, \texttt{rmdir}}
      \begin{itemize}
        \item \texttt{cp \textit{source} \textit{destination}} für copy: kopiert eine Datei
        \item \texttt{cp -r \textit{source} \textit{destination}}: kopiert ein Verzeichnis rekursiv
        \item \texttt{mv \textit{source} \textit{destination}} für move: verschiebt oder benennt eine Datei um
        \item \texttt{rm \textit{file}} für remove: löscht eine Datei
        \item \texttt{rm -r \textit{file}}: löscht ein Verzeichnis rekursiv
        \item \texttt{rmdir \textit{directory}} für remove directory: löscht ein leeres Verzeichnis
      \end{itemize}
    \end{frame}

    \begin{frame}{\texttt{cat}, \texttt{less}, \texttt{grep}}
      \begin{itemize}
        \item \texttt{cat [\textit{file}]} für concatenate: gibt den Inhalt einer (oder mehrerer) Dateien aus
        \item \texttt{less [\textit{file}]} (besser als \texttt{more}): zeigt eine Datei in einer navigablen Form an
        \item \texttt{grep \textit{pattern} [\textit{file}]} für ???: sucht nach einem Muster
        \item \texttt{grep -i \textit{pattern} [\textit{file}]}: ignoriert Groß- und Kleinschreibung
        \item \texttt{grep -r \textit{pattern} \textit{directory}}: sucht rekursiv in allen Dateien
        \item \texttt{locate \textit{pattern}}: sucht nach einem Muster in den Dateinamen des Dateisystems
      \end{itemize}
    \end{frame}

  \subsection{Nützliche Shell Features}
    \begin{frame}{Tastaturkürzel}
      \begin{itemize}
        \item \texttt{Ctrl-C}: beendet das laufende Programm
        \item \texttt{Ctrl-D}: EOF (end of file) eingeben, kann Programme beenden
        \item \texttt{Ctrl-L}: leert den Bildschirm
      \end{itemize}
    \end{frame}

    \begin{frame}{Ein- und Ausgabe}
      \begin{itemize}
        \item Programme haben einen Eingabe- und einen Ausgabestream
        \item diese können verschieden verwendet werden
        \item \texttt{\textit{command} > \textit{file}}: überschriebt eine Datei mit der Ausgabe
        \item \texttt{\textit{command} >> \textit{file}}: wie \texttt{>}, aber schreibt ans Ende der Datei, statt zu überschreiben
        \item \texttt{\textit{command} < \textit{file}}: verwendet eine Datei als Eingabe
        \item \texttt{\textit{command1} | \textit{command2}}: verwendet die Ausgabe eines Programms als Eingabe eines anderen
      \end{itemize}
    \end{frame}

    \begin{frame}{Globbing (\texttt{*})}
      \begin{itemize}
        \item \texttt{*} wird ersetzt durch alle passenden Dateien
      \end{itemize}
    \end{frame}
