\section{Unix Shell}
  \begin{frame}{Unix Shell}
    \tableofcontents[sectionstyle=show/hide,
                     hideothersubsections]
  \end{frame}

  \subsection{Dateisystem}
    \begin{frame}{Dateisystem}
      \begin{itemize}
        \item \texttt{/} trennt Teile eines Pfads (Verzeichnisse und Dateien) (Windows: \texttt{\textbackslash})
        \item Das gesamte Dateisystem bildet \emph{einen} Baum, beginnend mit der Wurzel \texttt{/}.
        \item Es gibt immer ein aktuelles Verzeichnis (working directory)
        \item Pfade können absolut (beginnend mit \texttt{/}) oder relativ zum aktuellen Verzeichnis angegeben werden
        \item drei spezielle Verzeichnisse:
          \begin{itemize}
            \item \texttt{.} das aktuelle Verzeichnis (oder der aktuelle im bisherigen Pfad, \texttt{a/./}~=~\texttt{a/})
            \item \texttt{..} das Oberverzeichnis des aktuellen Verzeichnisses (\texttt{a/b/../}~=~\texttt{a/})
            \item \texttt{\textasciitilde} das Heimverzeichnis (nur am Anfang eines Pfads)
          \end{itemize}
        \item Dateien, die mit \texttt{.} anfangen sind versteckt (z.B. \texttt{\textasciitilde/.vimrc}) und heißen Dotfiles
      \end{itemize}
    \end{frame}

  \subsection{Aufbau einer Eingabe}
    \begin{frame}{Aufbau einer Eingabe}
      \texttt{\$ ls -l --all \textit{directory}\\
              \textit{output}\\
              \$}
      \begin{center}
        \begin{tabular}{>{\tt}l l}
          \toprule
          \$                 & Prompt       \\
          ls                 & Befehl       \\
          -l                 & kurze Option \\
          --all              & lange Option \\
          \textit{directory} & Argument     \\
          \textit{output}    & Ausgabe      \\
          \bottomrule
        \end{tabular}
      \end{center}
    \end{frame}

    \begin{frame}
      \begin{itemize}
        \item \texttt{\$} ist nur ein Beispiel für einen Prompt, häufig wird das aktuelle Verzeichnis und/oder andere Informationen angezeigt
        \item kurze Optionen können zusammengefasst werden (\texttt{ls~-la} = \texttt{ls -l -a} = \texttt{ls -l --all})
        \item die Reihenfolge der Optionen ist egal
        \item meistens werden mehrere Argumente (z.B. Dateien) akzeptiert
      \end{itemize}
    \end{frame}

  \subsection{Befehle}
    \begin{frame}{\texttt{man}}
      \begin{itemize}
        \item \texttt{man \textit{topic}} für manual: zeigt die Hilfe für ein Programm
        \item Beispiel: \texttt{man man}
      \end{itemize}
    \end{frame}

    \begin{frame}{\texttt{pwd}, \texttt{cd}}
      \begin{itemize}
        \item \texttt{pwd} für print working directory: zeigt das aktuelle Verzeichnis
        \item \texttt{cd \textit{directory}} für change directory: wechselt in das angegebene Verzeichnis
        \item Beispiel:\\
          \texttt{\$ pwd\\
                  /home/ismo\\
                  \$ cd ../../etc\\
                  \$ pwd\\
                  /etc\\
                  \$ cd \textasciitilde} (\texttt{cd} reicht auch für das Home-Verzeichnis) \\
          \texttt{\$ pwd\\
                  /home/ismo}
      \end{itemize}
    \end{frame}

    \begin{frame}{ls}
      \begin{itemize}
        \item \texttt{ls [\textit{directory}]} für list: zeigt den Inhalt eines Verzeichnisses an
        \item \texttt{ls -l}: zeigt mehr Informationen über Dateien und Verzeichnisse
        \item \texttt{ls -a}: zeigt auch versteckte Dateien
        \item \texttt{ls -R}: zeigt auch den Inhalt von Unterverzeichnissen
        \item alle Optionen können kombiniert werden
      \end{itemize}
    \end{frame}

    \begin{frame}
      Beispiel:\\ 
      \texttt{\$ ls\\
              a/  b\\
              \$ ls -l\\
              total 4.0K\\
              drwxr-xr-x 2 ismo users 4.0K Sep 15 19:52 a/\\
              -rw-r--r-- 1 ismo users \ \ \ 0 Sep 15 19:52 b\\
              \$ ls -a\\
              ./  ../  a/  b\\
              \$ ls -R\\
              .:\\
              a/  b\\
              ~\\
              ./a:\\
              c}
    \end{frame}

    \begin{frame}{\texttt{mkdir}, \texttt{touch}}
      \begin{itemize}
        \item \texttt{mkdir \textit{directory}} für make directory: erstellt ein neues Verzeichnis
        \item \texttt{mkdir -p \textit{directory}}: erstellt ein neues Verzeichnis und alle notwendigen Oberverzeichnisse
        \item \texttt{touch \textit{file}}: erstellt eine neue, leere Datei
      \end{itemize}
    \end{frame}

    \begin{frame}
      Beispiel:\\
      \texttt{\$ ls\\
              \$ mkdir a\\
              \$ mkdir b/c\\
              mkdir: cannot create directory ‘b/c’: No such file or directory\\
              \$ mkdir -p b/c\\
              \$ touch b/file\\
              \$ ls -R\\
              .:\\
              a/  b/\\
              ~\\
              ./a:\\
              ~\\                  
              ./b:\\
              c/ file\\
              ~\\
              ./b/c:}
    \end{frame}

    \begin{frame}{\texttt{cp}, \texttt{mv}, \texttt{rm}, \texttt{rmdir}}
      \begin{itemize}
        \item \texttt{cp \textit{source} \textit{destination}} für copy: kopiert eine Datei
        \item \texttt{cp -r \textit{source} \textit{destination}}: kopiert ein Verzeichnis rekursiv
        \item das Ziel kann ein Verzeichnis oder der exakte Pfad sein
        \item \texttt{\textit{source}} können mehrere Dateien sein, der letzte Pfad zählt als \texttt{\textit{destination}}
        \item \texttt{mv \textit{source} \textit{destination}} für move: verschiebt order benennt eine Datei um
        \item Ziel kann wir bei \texttt{cp} sein
        \item \texttt{rm \textit{file}} für remove: löscht eine Datei
        \item \texttt{rm -r \textit{file}}: löscht ein Verzeichnis rekursiv
        \item \texttt{rmdir \textit{directory}} für remove directory: löscht ein leeres Verzeichnis
        \item \texttt{rm -r} kann statt \texttt{rmdir} verwendet werden
      \end{itemize}
    \end{frame}

    \begin{frame}
      Beispiel:\\
      \texttt{\$ ls\\
              a\\
              \$ cp a b\\
              \$ ls\\
              a  b\\
              \$ mv b c\\
              \$ ls\\
              a  c\\
              \$ rm a\\
              \$ ls\\
              c}
    \end{frame}

    \begin{frame}{\texttt{cat}, \texttt{less}, \texttt{grep}, \texttt{echo}}
      \begin{itemize}
        \item \texttt{cat [\textit{file}]} für concatenate: gibt den Inhalt einer (oder mehrerer) Dateien aus
        \item \texttt{less [\textit{file}]} (besser als \texttt{more}): zeigt eine Datei in einer navigablen Form an
        \item \texttt{grep \textit{pattern} [\textit{file}]} für ???: sucht nach einem Muster
        \item \texttt{grep -i \textit{pattern} [\textit{file}]}: ignoriert Groß- und Kleinschreibung
        \item \texttt{grep -r \textit{pattern} \textit{directory}}: sucht rekursiv in allen Dateien
        \item \texttt{echo \textit{text}}: gibt einen Text aus
      \end{itemize}
    \end{frame}

  \subsection{Nützliche Shell Features}
    \begin{frame}{Tastaturkürzel}
      \begin{itemize}
        \item \texttt{Ctrl-C}: beendet das laufende Programm
        \item \texttt{Ctrl-D}: EOF (end of file) eingeben, kann Programme beenden
        \item \texttt{Ctrl-L}: leert den Bildschirm
      \end{itemize}
    \end{frame}

    \begin{frame}{Ein- und Ausgabe}
      \begin{itemize}
        \item >, >>, <
        \item pipes
      \end{itemize}
    \end{frame}

    \begin{frame}{Globbing}
      \begin{itemize}
        \item glob (*, ...)
      \end{itemize}
    \end{frame}
