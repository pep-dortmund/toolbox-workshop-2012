  \section{Unix Shell}
    \subsection{Allgemeines}
      \begin{frame}{Dateisystem}
        \begin{itemize}
          \item \texttt{/} trennt Teile eines Pfads (Ordner und Dateien) (Windows: \texttt{\textbackslash})
          \item Das gesamte Dateisystem bildet \emph{einen} Baum, beginnend mit \texttt{/}
          \item Es gibt immer einen aktuellen Ordner (working directory)
          \item Pfade können absolut (beginnend mit \texttt{/}) oder relativ zum aktuellen Ordner angegeben werden
          \item drei spezielle Ordner:
            \begin{itemize}
              \item \texttt{.} der aktuelle Ordner (oder der aktuelle im bisherigen Pfad, \texttt{a/./}~=~\texttt{a/})
              \item \texttt{..} der Oberordner des aktuellen Ordners (\texttt{a/b/../}~=~\texttt{a/})
              \item \texttt{\textasciitilde} das Heimverzeichnis (nur am Anfang eines Pfads)
            \end{itemize}
          \item Dateien, die mit \texttt{.} anfangen sind versteckt (z.B. \texttt{\textasciitilde/.vimrc})
        \end{itemize}
      \end{frame}

      \begin{frame}{Aufbau einer Eingabe}
        \texttt{\$ ls -l --all \textit{directory}\\
                \textit{output}\\
                \$}
        \begin{center}
          \begin{tabular}{>{\tt}l l}
            \toprule
            \$                 & Prompt       \\
            ls                 & Befehl       \\
            -l                 & kurze Option \\
            --all              & lange Option \\
            \textit{directory} & Argument     \\
            \textit{output}    & Ausgabe      \\
            \bottomrule
          \end{tabular}
        \end{center}
      \end{frame}

      \begin{frame}
        \begin{itemize}
          \item \texttt{\$} ist nur ein Beispiel für einen Prompt, häufig wird das aktuelle Verzeichnis und/oder andere Informationen angezeigt
          \item kurze Optionen können zusammengefasst werden (\texttt{ls~-la} = \texttt{ls -l -a} = \texttt{ls -l --all})
          \item die Reihenfolge der Optionen ist egal
          \item meistens werden mehrere Argumente (z.B. Dateien) akzeptiert
        \end{itemize}
      \end{frame}

    \subsection{Befehle}
      \begin{frame}{\texttt{man}}
        \begin{itemize}
          \item \texttt{man} für manual
          \item zeigt die Hilfe für ein Programm
          \item Beispiel: \texttt{man man}
        \end{itemize}
      \end{frame}

      \begin{frame}{\texttt{pwd}, \texttt{cd}}
        \begin{itemize}
          \item \texttt{pwd} für print working directory: zeigt das aktuelle Verzeichnis
          \item \texttt{cd} für change directory: wechselt in das angegebene Verzeichnis
          \item Beispiel:\\
            \texttt{\$ pwd\\
                    /home/ismo\\
                    \$ cd ../../etc\\
                    \$ pwd\\
                    /etc\\
                    \$ cd \textasciitilde\\
                    \$ pwd\\
                    /home/ismo}
        \end{itemize}
      \end{frame}

      \begin{frame}{ls}
        \begin{itemize}
          \item \texttt{ls} für list: zeigt den Inhalt eines Verzeichnisses an
          \item \texttt{ls -l}: zeigt mehr Informationen über Dateien und Verzeichnisse
          \item \texttt{ls -a}: zeigt auch versteckte Dateien
          \item \texttt{ls -R}: zeigt auch den Inhalt von Unterordnern
          \item alle Optionen können kombiniert werden
        \end{itemize}
      \end{frame}

      \begin{frame}
        Beispiel:\\
        \texttt{\$ ls\\
                a/  b\\
                \$ ls -l\\
                total 4.0K\\
                drwxr-xr-x 2 ismo users 4.0K Sep 15 19:52 a/\\
                -rw-r--r-- 1 ismo users \ \ \ 0 Sep 15 19:52 b\\
                \$ ls -a\\
                ./  ../  a/  b\\
                \$ ls -R\\
                .:\\
                a/  b\\
                ~\\
                ./a:\\
                c}
      \end{frame}

      \begin{frame}{\texttt{mkdir}}
      \end{frame}

      \begin{frame}{cp, mv, rm}
        \begin{itemize}
          \item cp -r\\
          \item rm -r\\
          \item rm -f
        \end{itemize}
      \end{frame}

      \begin{frame}{cat, less}
      \end{frame}

      \begin{frame}{grep}
      \end{frame}

      \begin{frame}{Allgemeines}
        \begin{itemize}
          \item pipes\\
          \item ctrl-c\\
          \item ctrl-d\\
          \item .., .\\
          \item >, >>, <\\
          \item glob (*, ...)
        \end{itemize}
      \end{frame}