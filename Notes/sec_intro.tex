\chapter{Einleitung}
\section{Motivation}
\section{PeP et al.}

\section{Installation}
\subsection{Ubuntu 12.04}
Die benötigten Programme und Libraries kann man unter Ubuntu per Kommandozeile installieren.

\subsubsection{Git}
Die Versionskontrolle 'Git' installiert man mit dem Befehl:
\begin{verbatim}
$ sudo apt-get install git
\end{verbatim}

\subsubsection{Python und Libraries}
Zusätzlich zu Python 3 sollte man die Pythonbibliotheken NumPy und SciPy, die wissenschaftliches Rechnen deutlich vereinfachen, installieren.
Bei iPython handelt es sich um eine interaktive Konsole für Python und seine Bibliotheken, mit der man, ähnlich wie mit der Kommandozeile, Befehle oder Skripte direkt ausführen kann.
Der Installationsbefehl lautet:
\begin{verbatim}
$ sudo apt-get install python3 python3-numpy python3-scipy ipython3
\end{verbatim}

\subsubsection{Matplotlib}
Leider gibt es noch kein Paket für die Python3-Version von Matplotlib, dem Plotprogramm für Python.
Der Sourcecode der Bibliothek kann aber so heruntergeladen und kompiliert werden:
\begin{verbatim}
$ sudo apt-get install python3-dev libpng-dev libfreetype6-dev python3-pyqt4
$ wget https://github.com/matplotlib/matplotlib/matplotlib-1.2.0rc1.tar.gz
$ tar xzvf matplotlib-1.2.0rc1.tar.gz
$ cd matplotlib-1.2.0rc1
$ python3 setup.py build
$ sudo python3 setup.py install
$ mkdir -p ~/.matplotlib && echo 'backend: Qt4Agg' > ~/.matplotlib/matplotlibrc
\end{verbatim}

\subsubsection{Test}
Zum Schluss kann man Matplotlib noch testen:
\begin{verbatim}
$ ipython3 --pylab
In[1]: plot([1, 2, 3])
\end{verbatim}
