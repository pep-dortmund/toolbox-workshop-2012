\chapter{Einleitung}
\section{Motivation}
\url{http://www.americanscientist.org/issues/pub/wheres-the-real-bottleneck-in-scientific-computing}
\section{PeP et al.}

\section{Installation}
Ein \verb|\| am Ende einer Eingabezeile bedeutet, dass die Zeile aus Platzgründen gebrochen wurde.
Man sollte das \verb|\| weglassen und die Zeile ganz, ohne Umbruch eingeben.
Der Abschnitt \ref{install-test} gilt für alle Unix-artigen Betriebssysteme, also sowohl Linux als auch Mac OS X.
\subsection{Windows 7}
\subsubsection{Downloads}
Ein Archiv mit den benötigten Installationsdateien kann unter \\
\url{http://files.babushk.in/toolbox.zip} \\
heruntergeladen werden.

\subsubsection{Git}
\begin{itemize}
  \item \texttt{Git-1.7.11-preview20120710.exe}
  \item Pfad: \verb|C:\Programs\Git|
  \item Use a TrueType font in all console windows
\end{itemize}

\subsubsection{Make}
Diese Dateien in \verb|C:\Programs\Git\bin| kopieren:
\begin{itemize}
  \item \texttt{make.exe}
  \item \texttt{msys-iconv-2.dll}
  \item \texttt{msys-intl-8.dll}
\end{itemize}

\subsubsection{KDiff3}
\begin{itemize}
  \item \texttt{KDiff3-32bit-Setup\_0.9.97.exe}
  \item Pfad: \verb|C:\Programs\KDiff3|
\end{itemize}

\subsubsection{Python}
\begin{itemize}
  \item \texttt{python-3.2.3.msi}
  \item Pfad: \verb|C:\Programs\Python32|
\end{itemize}

\subsubsection{Numpy, Scipy, Matplotlib, Distribute, Pyreadline, Pyzmq, PyQt4, Pygments}
\begin{itemize}
  \item \texttt{numpy-1.6.2-win32-superpack-python3.2.exe}
  \item \texttt{scipy-0.10.1-win32-superpack-python3.2.exe}
  \item \texttt{matplotlib-1.2.0rc2.win32-py3.2.exe}
  \item \texttt{distribute-0.6.28.win32-py3.2.exe}
  \item \texttt{pyreadline-2.0-dev1.win32-py3.2.exe}
  \item \texttt{pyzmq-2.2.0.win32-py3.2.msi}
  \item \texttt{PyQt-Py3.2-x86-gpl-4.9.4-1.exe}
  \item \texttt{Pygments-1.5.win32-py3.2.exe}
\end{itemize}

\subsubsection{IPython}
\begin{itemize}
  \item \texttt{ipython-0.13.py3-win32.exe}
  \item Rechstklick $\rightarrow$ Als Administrator ausführen
\end{itemize}

\subsubsection{Letzte Schritte}
\begin{itemize}
  \item Start $\rightarrow$ suche cmd.exe $\rightarrow$ Rechtsklick $\rightarrow$ Als Administrator ausführen
  \item
\begin{verbatim}
setx /M PATH "C:\Programs\Python32;%PATH%"
copy C:\Programs\Python32\python.exe C:\Programs\Python32\python3.exe
\end{verbatim}
\end{itemize}
Jetzt müssen Git und Matplotlib noch eingestellt werden.
In Git Bash ausführen (richtige Daten eintragen):
\begin{verbatim}
git config --global user.email "ismo.toijala@udo.edu"
git config --global user.name "Ismo Toijala"
mkdir -p ~/.matplotlib
echo 'backend : Qt4Agg' > ~/.matplotlib/matplotlibrc
echo "
[merge]
    tool = kdiff3
    
[mergetool "kdiff3"]
    path = C:/Programs/KDiff3/kdiff3.exe
    keepBackup = false
    trustExitCode = false" >> ~/.gitconfig
\end{verbatim}

\subsubsection{Test}
Jetzt kann man noch testen, ob alles vernünftig funktioniert:
\begin{itemize}
  \item Start $\rightarrow$ Programme $\rightarrow$ IPython (Py3.2 32 bit) $\rightarrow$ pylab
  \item \texttt{plot([1, 2, 3])}
\end{itemize}
Nach dieser Eingabe sollte ein Plot erscheinen.

\subsection{Mac OS X}
Am einfachsten ist die Installation bei OS X unter Verwendung von Mac Ports.
Auf der Website \url{http://www.macports.org/} gibt es ein entsprechendes .dmg.
Dabei ist zu beachten, dass XCode installiert sein muss. Sollte kein XCode vorhanden sein, ist es auf der OS X Installations-DVD zu finden oder kann über den AppStore heruntergeladen werden (OS X 10.7 vorausgesetzt).

Ist Mac Ports erst einmal installiert, ist die Installation der benötigten Programme und Bibliotheken denkbar einfach.
In der Kommandozeile (z.B. Terminal.app) wird einfach
\begin{verbatim}
sudo port install gmake git-core python27 py27-ipython py27-numpy \
py27-scipy py27-matplotlib kdiff3 py27-pyqt4 py27-sip py27-pygments
\end{verbatim}
eingegeben.
Das Ausführen von
\begin{verbatim}
sudo port select python python27
\end{verbatim}
sorgt dafür, dass die neu installierte Python-Version als Standard verwendet wird und nicht die von Apple mitgelieferte.

\subsection{Ubuntu 12.04}
Die benötigten Programme und Libraries kann man unter Ubuntu am schnellsten per Kommandozeile installieren.

\subsubsection{Git}
Die Versionskontrolle 'Git' installiert man mit dem Befehl
\begin{verbatim}
sudo apt-get install make git kdiff3
\end{verbatim}

\subsubsection{Python und Bibliotheken}
Zusätzlich zu Python 3 sollte man die Python-Bibliotheken NumPy und SciPy (für wissenschaftliche Berechnungen) installieren.
Bei IPython handelt es sich um eine interaktive Konsole für Python, mit der man, ähnlich wie mit der Kommandozeile, Befehle oder Skripte ausführen kann.
Der Installationsbefehl lautet
\begin{verbatim}
sudo apt-get install python3 python3-numpy python3-scipy ipython3 \
python3-pyqt4 python3-zmq python3-pygments python3-sip
\end{verbatim}

\subsubsection{Matplotlib}
Leider gibt es noch kein Paket für die Python3-Version von Matplotlib (zum Erstellen von Plots).
Es gibt aber ein externes Repository mit der neuesten Version:
\begin{verbatim}
sudo add-apt-repository ppa:takluyver/matplotlib-daily
sudo apt-get update
sudo apt-get install python3-matplotlib
\end{verbatim}

\subsection{Arch Linux}
\begin{verbatim}
sudo pacman -S make git python python-numpy python-scipy \
ipython pyqt sip python-pygments python-pyzmq kdiff3
yaourt -S python-matplotlib-git
\end{verbatim}

\subsection{Einstellungen und Testen}
\label{install-test}
Git und Matplotlib einstellen (richtige Daten eintragen):
\begin{verbatim}
git config --global user.email "ismo.toijala@udo.edu"
git config --global user.name "Ismo Toijala"
mkdir -p ~/.matplotlib
echo 'backend : Qt4Agg' > ~/.matplotlib/matplotlibrc
echo "
[merge]
    tool = kdiff3" >> ~/.gitconfig
\end{verbatim}
Jetzt kann man noch testen, ob alles vernünftig funktioniert:
\begin{verbatim}
ipython3 --pylab
plot([1,2,3])
\end{verbatim}
Nach dieser Eingabe sollte ein Plot erscheinen.
