\documentclass[a4paper,11pt,oneside]{scrbook}

% Font
\usepackage{fontspec}
\setmainfont{Latin Modern Roman}
\setsansfont{Latin Modern Sans}
\setmonofont{Latin Modern Mono}

% Language
\usepackage[ngerman]{babel}

% Layout
\usepackage{fancyhdr}
\renewcommand{\headrulewidth}{0.8pt}
\renewcommand{\footrulewidth}{0.4pt}
\renewcommand{\headsep}{32pt}
\renewcommand{\headheight}{42pt}
\usepackage[margin=10pt,font=small,labelfont=bf]{caption}

% Tables
\usepackage{booktabs}
\usepackage{array}

% Color (mostly needed for code listings)
\usepackage[usenames,dvipsnames]{xcolor}

% Graphics
\usepackage{graphicx}

% References
\usepackage{varioref}
\usepackage[breaklinks,unicode=true,pdfcreator={},pdfproducer={}]{hyperref}
\urlstyle{tt}
\AtBeginDocument{\renewcommand{\subsectionautorefname}{Abschnitt}}
\AtBeginDocument{\renewcommand{\subsubsectionautorefname}{Abschnitt}}

% Maths
\usepackage{amsfonts}
\usepackage{amssymb}
\usepackage{amsmath}
\usepackage{amscd}
\usepackage{amstext}

% Code listings
\usepackage{minted}

\newcommand{\mytitle}[1]{
    % Fancy headlines
    \pagestyle{fancy}
    \fancyhead[R]{\textit{\nouppercase{\leftmark}}}
    \fancyhead[L]{\textit{\nouppercase{\rightmark}}}
    \fancyfoot[R]{\thepage}
    \cfoot{}
    % PDF Information
    \hypersetup{
        pdfsubject={PeP et al. Toolbox Workshop},
        pdftitle={Toolbox Workshop},
        pdfauthor={Igor Babuschkin, Kevin Dungs, Ismo Toijala},
    }

    \begin{titlepage}
        \begin{center}
                \Huge\color{Red}{\textbf{DRAFT}}
            \end{center}

        \begin{flushright}
            \rule{\linewidth}{1.625mm} \\[1cm]
            
            \Large
                \textbf{PeP et al.} \\[0.3125cm]
            \normalsize

            \Huge
                \textbf{Toolbox Workshop}
            \normalsize \\[1.625cm]
        
            \Large
                TU Dortmund
            \normalsize \\
            #1 \\[.3125cm]
            
            \rule{\linewidth}{0.625mm} \\ [.3125cm]

            \begin{minipage}[t]{0.3\textwidth}\begin{flushright}
                \large Igor Babuschkin \\
                \small \href{mailto:igor.babuschkin@udo.edu}{igor.babuschkin@udo.edu}
            \end{flushright}\end{minipage}
            \begin{minipage}[t]{0.3\textwidth}\begin{flushright}
                \large Kevin Dungs \\
                \small \href{mailto:kevin.dungs@udo.edu}{kevin.dungs@udo.edu}
            \end{flushright}\end{minipage}
            \begin{minipage}[t]{0.3\textwidth}\begin{flushright}
                \large Ismo Toijala \\
                \small \href{mailto:ismo.toijala@udo.edu}{ismo.toijala@udo.edu}
            \end{flushright}\end{minipage}

        \end{flushright}
    \end{titlepage}
}


\begin{document}
\mytitle{\today}
\newpage
\tableofcontents

\chapter{Einleitung}
\section{Motivation}
\section{Vorgehen}
\section{PeP et al.}

\section{Installation}

\subsection{Ubuntu 12.04}
Die benötigten Programme und Libraries kann man unter Ubuntu per Kommandozeile installieren.

\subsubsection{Git installieren}
\begin{verbatim}
$ sudo apt-get install git
\end{verbatim}

\subsubsection{Python und Libraries installieren}
\begin{verbatim}
$ sudo apt-get install python3 python3-numpy python3-scipy ipython3
\end{verbatim}

\subsubsection{Matplotlib kompilieren}

Leider gibt es noch kein Paket für die Python3-Version von Matplotlib.
Die Bibliothek kann aber so kompiliert werden:

\begin{verbatim}
$ sudo apt-get install python3-dev libpng-dev libfreetype6-dev python3-pyqt4
$ wget https://github.com/matplotlib/matplotlib/matplotlib-1.2.0rc1.tar.gz
$ tar xzvf matplotlib-1.2.0rc1.tar.gz
$ cd matplotlib-1.2.0rc1
$ python3 setup.py build
$ sudo python3 setup.py install
$ mkdir -p ~/.matplotlib && echo 'backend: Qt4Agg' > ~/.matplotlib/matplotlibrc
\end{verbatim}

\subsubsection{Testen}

Zum Schluss kann man Matplotlib noch testen:

\begin{verbatim}
$ ipython3 --pylab
In[1]: plot([1, 2, 3])
\end{verbatim}

\chapter{*nix Grundlagen}
\section{Benutzung der Shell}
\section{SSH}


\chapter{Programmieren}
\section{Python}
\subsection{Grundlagen}
\subsection{Numpy}
\subsection{Scipy}
\subsection{Matplotlib}
\section{C++}


\chapter{Versionskontrolle}
\section{Grundlagen}
\section{Git}


\chapter{LaTeX}


\end{document}
