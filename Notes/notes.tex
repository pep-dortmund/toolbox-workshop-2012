% Font
\usepackage{lmodern}
%\usepackage{fourier}
\usepackage{fontspec}
\usepackage{textcomp}

% Language
\usepackage[ngerman]{babel}

% Layout
\usepackage{fancyhdr}
\renewcommand{\headrulewidth}{0.8pt}
\renewcommand{\footrulewidth}{0.4pt}
\renewcommand{\headsep}{32pt}
\renewcommand{\headheight}{42pt}
\usepackage[margin=10pt,font=small,labelfont=bf]{caption}
\usepackage{soulutf8}

% Tables
\usepackage{longtable}
\usepackage{booktabs}
\usepackage{float}

% Color (mostly needed for code listings)
\usepackage[usenames,dvipsnames]{xcolor}

% Graphics
\usepackage{graphicx}

% Literature
%\usepackage[style=numeric,backend=biber]{biblatex}
%\usepackage[notbib]{tocbibind}
%\addbibresource{../literature.bib}

% Units
\usepackage{siunitx}
\sisetup{
    locale=DE,
    separate-uncertainty=true,
    per-mode=fraction
}

% References
\usepackage{varioref}
\usepackage[breaklinks]{hyperref}
%\usepackage{hypdvips}
\AtBeginDocument{\renewcommand{\subsectionautorefname}{Abschnitt}}
\AtBeginDocument{\renewcommand{\subsubsectionautorefname}{Abschnitt}}

% URLs
\usepackage{url}
\urlstyle{tt}

% Maths
\usepackage{amsfonts}
\usepackage{amssymb}
\usepackage{amsmath}
\usepackage{amscd}
\usepackage{amstext}
\usepackage{tensor}
% Sets
\newcommand{\N}{\mathbb{N}}
\newcommand{\Z}{\mathbb{Z}}
\newcommand{\Q}{\mathbb{Q}}
\newcommand{\R}{\mathbb{R}}
\newcommand{\C}{\mathbb{C}}
% Identity
\newcommand{\E}{\mathbb{1}}
% Special numbers
\renewcommand{\i}{\mathbf{i}}
\newcommand{\e}{\mathbf{e} }
\newcommand{\enull}{\varepsilon_0}
% Special Symbols
\newcommand{\lagrange}{\mathcal{L}}
\newcommand{\landau}{\mathcal{O}}
% Operators
\newcommand{\diff}{\mathrm{d}}
\newcommand{\laplace}{\; \mathbf{\Delta}}
\newcommand{\rot}{\; \mathrm{rot} \,}
\newcommand{\grad}{\; \mathrm{grad} \,}
\newcommand{\dive}{\; \mathrm{div} \,}
\renewcommand{\Re}{\; \mathrm{Re} \,}
\renewcommand{\Im}{\; \mathrm{Im} \,}
% Trigonometry
\newcommand{\cotanh}{\; \mathrm{cotanh} \,}
\newcommand{\asinh}{\; \mathrm{areasinh} \,}
\newcommand{\acosh}{\; \mathrm{areacosh} \,}
\newcommand{\atanh}{\; \mathrm{areatanh} \,}
\newcommand{\acoth}{\; \mathrm{areacoth} \,}
% Absolute and Norm
\newcommand{\abs}[1]{\left\lvert#1\right\rvert}
\newcommand{\norm}[1]{\left\lVert#1\right\rVert}
% Mean values and standard deviation
\newcommand{\mean}[1]{\langle #1 \rangle}
\newcommand{\std}[1]{\sigma_{#1}}
% Shortcut for sums
\newcommand{\sumi}[1]{\sum\limits_{#1 = 0}^{\infty}}
\newcommand{\sumk}[2]{\sum\limits_{#1}^{#2}}
% Shortcut for limes
\newcommand{\limi}[1]{\lim\limits_{#1\rightarrow\infty}}
% Shortcut for integrals
\newcommand{\inti}{\int\limits_{-\infty}^\infty}
\newcommand{\intx}[2]{\int\limits_{#1}^{#2}}
% Shortcut for derivatives
\newcommand{\deriv}[2]{\frac{\diff #1}{\diff #2}}
\newcommand{\derivsq}[2]{\frac{\diff^2 #1}{\diff #2^2}}
\newcommand{\partderiv}[2]{\frac{\partial #1}{\partial #2}}
\newcommand{\partderivsq}[2]{\frac{\partial^2 #1}{\partial #2^2}}
% Dirac braket notation
\newcommand{\bra}[1]{\langle #1 \mid}
\newcommand{\ket}[1]{\mid #1 \rangle}
\newcommand{\braket}[2]{\langle #1 \mid #2 \rangle}
% Miscellaneous
\newcommand{\qed}{\hfill $\square$\\}
\newcommand{\me}{\overset{!}{=}}

% Equations
\usepackage{IEEEtrantools}
\newenvironment{eqn}{\begin{IEEEeqnarray}{c}}{\end{IEEEeqnarray}}
\newenvironment{eqn*}{\begin{IEEEeqnarray*}{c}}{\end{IEEEeqnarray*}}
\newenvironment{eqns}[1]{\begin{IEEEeqnarray}{#1}}{\end{IEEEeqnarray}}
\newenvironment{eqns*}[1]{\begin{IEEEeqnarray*}{#1}}{\end{IEEEeqnarray*}}
\IEEEeqnarraydefcolsep{0}{\leftmargini} % for well-aligned text in equation arrays %

% Code listings
\usepackage{minted}

% Abbreviations for repeated commands
\newcommand{\img}[4]{
    \begin{figure}[H]
        \begin{center}
            \caption{#4}
            \includegraphics[width=#1]{#2}
            \label{#3}
        \end{center}
    \end{figure}
}

\newcommand{\mytitle}[1]{
    % Fancy headlines
    \pagestyle{fancy}
    \fancyhead[RE]{\textit{\nouppercase{\leftmark}}}
    \fancyhead[LO]{\textit{\nouppercase{\rightmark}}}
    \fancyfoot[LE,RO]{\thepage}
    \cfoot{}
    % PDF Information
    \hypersetup{
        pdfsubject={PeP et al. Toolbox Workshop},
        pdftitle={Toolbox Workshop},
        pdfauthor={Igor Babuschkin, Kevin Dungs, Ismo Toijala},
    }

    \begin{titlepage}
        \begin{center}
                \Huge\color{Red}{\textbf{DRAFT}}
            \end{center}

        \begin{flushright}
            \rule{\linewidth}{1.625mm} \\[1cm]
            
            \Large
                \textbf{PeP et al.} \\[0.3125cm]
            \normalsize

            \Huge
                \textbf{Toolbox Workshop}
            \normalsize \\[1.625cm]
        
            \Large
                TU Dortmund
            \normalsize \\
            #1 \\[.3125cm]
            
            \rule{\linewidth}{0.625mm} \\ [.3125cm]

            \begin{minipage}[t]{0.3\textwidth}\begin{flushright}
                \large Igor Babuschkin \\
                \small \href{mailto:igor.babuschkin@udo.edu}{igor.babuschkin@udo.edu}
            \end{flushright}\end{minipage}
            \begin{minipage}[t]{0.3\textwidth}\begin{flushright}
                \large Kevin Dungs \\
                \small \href{mailto:kevin.dungs@udo.edu}{kevin.dungs@udo.edu}
            \end{flushright}\end{minipage}
            \begin{minipage}[t]{0.3\textwidth}\begin{flushright}
                \large Ismo Toijala \\
                \small \href{mailto:ismo.toijala@udo.edu}{ismo.toijala@udo.edu}
            \end{flushright}\end{minipage}

        \end{flushright}
    \end{titlepage}
}


\begin{document}
\mytitle{\today}
\newpage
\tableofcontents

\chapter{Einleitung}
\section{Motivation}
\section{Vorgehen}
\section{PeP et al.}

\section{Installation}

\subsection{Ubuntu 12.04}
Die benötigten Programme und Libraries kann man unter Ubuntu per Kommandozeile installieren.

\subsubsection{Git}
Die Versionskontrolle 'Git' installiert man mit dem Befehl:
\begin{verbatim}
$ sudo apt-get install git
\end{verbatim}

\subsubsection{Python und Libraries}
Zusätzlich zu Python 3 sollte man die Pythonbibliotheken NumPy und SciPy, die wissenschaftliches Rechnen deutlich vereinfachen, installieren.
Bei iPython handelt es sich um eine interaktive Konsole für Python und seine Bibliotheken, mit der man, ähnlich wie mit der Kommandozeile, Befehle oder Skripte direkt ausführen kann.
Der Installationsbefehl lautet:
\begin{verbatim}
$ sudo apt-get install python3 python3-numpy python3-scipy ipython3
\end{verbatim}

\subsubsection{Matplotlib}
Leider gibt es noch kein Paket für die Python3-Version von Matplotlib, dem Plotprogramm für Python.
Der Sourcecode der Bibliothek kann aber so heruntergeladen und kompiliert werden:
\begin{verbatim}
$ sudo apt-get install python3-dev libpng-dev libfreetype6-dev python3-pyqt4
$ wget https://github.com/matplotlib/matplotlib/matplotlib-1.2.0rc1.tar.gz
$ tar xzvf matplotlib-1.2.0rc1.tar.gz
$ cd matplotlib-1.2.0rc1
$ python3 setup.py build
$ sudo python3 setup.py install
$ mkdir -p ~/.matplotlib && echo 'backend: Qt4Agg' > ~/.matplotlib/matplotlibrc
\end{verbatim}

\subsubsection{Test}
Zum Schluss kann man Matplotlib noch testen:
\begin{verbatim}
$ ipython3 --pylab
In[1]: plot([1, 2, 3])
\end{verbatim}

\section{Unix Shell}
  \begin{frame}{Unix Shell}
    \tableofcontents[sectionstyle=show/hide,
                     hideothersubsections]
  \end{frame}

  \subsection{Dateisystem}
    \begin{frame}{Dateisystem}
      \begin{itemize}
        \item \texttt{/} trennt Teile eines Pfads (Verzeichnisse und Dateien) (Windows: \texttt{\textbackslash})
        \item Das gesamte Dateisystem bildet \emph{einen} Baum, beginnend mit der Wurzel \texttt{/}.
        \item Es gibt immer ein aktuelles Verzeichnis (working directory)
        \item Pfade können absolut (beginnend mit \texttt{/}) oder relativ zum aktuellen Verzeichnis angegeben werden
        \item drei spezielle Verzeichnisse:
          \begin{itemize}
            \item \texttt{.} das aktuelle Verzeichnis (oder der aktuelle im bisherigen Pfad, \texttt{a/./}~=~\texttt{a/})
            \item \texttt{..} das Oberverzeichnis des aktuellen Verzeichnisses (\texttt{a/b/../}~=~\texttt{a/})
            \item \texttt{\textasciitilde} das Heimverzeichnis (nur am Anfang eines Pfads)
          \end{itemize}
        \item Dateien, die mit \texttt{.} anfangen sind versteckt (z.B. \texttt{\textasciitilde/.vimrc}) und heißen Dotfiles
      \end{itemize}
    \end{frame}

  \subsection{Aufbau einer Eingabe}
    \begin{frame}{Aufbau einer Eingabe}
      \texttt{\$ ls -l --all \textit{directory}\\
              \textit{output}\\
              \$}
      \begin{center}
        \begin{tabular}{>{\tt}l l}
          \toprule
          \$                 & Prompt       \\
          ls                 & Befehl       \\
          -l                 & kurze Option \\
          --all              & lange Option \\
          \textit{directory} & Argument     \\
          \textit{output}    & Ausgabe      \\
          \bottomrule
        \end{tabular}
      \end{center}
    \end{frame}

    \begin{frame}
      \begin{itemize}
        \item \texttt{\$} ist nur ein Beispiel für einen Prompt, häufig wird das aktuelle Verzeichnis und/oder andere Informationen angezeigt
        \item kurze Optionen können zusammengefasst werden (\texttt{ls~-la} = \texttt{ls -l -a} = \texttt{ls -l --all})
        \item die Reihenfolge der Optionen ist egal
        \item meistens werden mehrere Argumente (z.B. Dateien) akzeptiert
      \end{itemize}
    \end{frame}

  \subsection{Befehle}
    \begin{frame}{\texttt{man}}
      \begin{itemize}
        \item \texttt{man \textit{topic}} für manual: zeigt die Hilfe für ein Programm
        \item Beispiel: \texttt{man man}
      \end{itemize}
    \end{frame}

    \begin{frame}{\texttt{pwd}, \texttt{cd}}
      \begin{itemize}
        \item \texttt{pwd} für print working directory: zeigt das aktuelle Verzeichnis
        \item \texttt{cd \textit{directory}} für change directory: wechselt in das angegebene Verzeichnis
        \item Beispiel:\\
          \texttt{\$ pwd\\
                  /home/ismo\\
                  \$ cd ../../etc\\
                  \$ pwd\\
                  /etc\\
                  \$ cd \textasciitilde} (\texttt{cd} reicht auch für das Home-Verzeichnis) \\
          \texttt{\$ pwd\\
                  /home/ismo}
      \end{itemize}
    \end{frame}

    \begin{frame}{ls}
      \begin{itemize}
        \item \texttt{ls [\textit{directory}]} für list: zeigt den Inhalt eines Verzeichnisses an
        \item \texttt{ls -l}: zeigt mehr Informationen über Dateien und Verzeichnisse
        \item \texttt{ls -a}: zeigt auch versteckte Dateien
        \item \texttt{ls -R}: zeigt auch den Inhalt von Unterverzeichnissen
        \item alle Optionen können kombiniert werden
      \end{itemize}
    \end{frame}

    \begin{frame}
      Beispiel:\\ 
      \texttt{\$ ls\\
              a/  b\\
              \$ ls -l\\
              total 4.0K\\
              drwxr-xr-x 2 ismo users 4.0K Sep 15 19:52 a/\\
              -rw-r--r-- 1 ismo users \ \ \ 0 Sep 15 19:52 b\\
              \$ ls -a\\
              ./  ../  a/  b\\
              \$ ls -R\\
              .:\\
              a/  b\\
              ~\\
              ./a:\\
              c}
    \end{frame}

    \begin{frame}{\texttt{mkdir}, \texttt{touch}}
      \begin{itemize}
        \item \texttt{mkdir \textit{directory}} für make directory: erstellt ein neues Verzeichnis
        \item \texttt{mkdir -p \textit{directory}}: erstellt ein neues Verzeichnis und alle notwendigen Oberverzeichnisse
        \item \texttt{touch \textit{file}}: erstellt eine neue, leere Datei
      \end{itemize}
    \end{frame}

    \begin{frame}
      Beispiel:\\
      \texttt{\$ ls\\
              \$ mkdir a\\
              \$ mkdir b/c\\
              mkdir: cannot create directory ‘b/c’: No such file or directory\\
              \$ mkdir -p b/c\\
              \$ touch b/file\\
              \$ ls -R\\
              .:\\
              a/  b/\\
              ~\\
              ./a:\\
              ~\\                  
              ./b:\\
              c/ file\\
              ~\\
              ./b/c:}
    \end{frame}

    \begin{frame}{\texttt{cp}, \texttt{mv}, \texttt{rm}, \texttt{rmdir}}
      \begin{itemize}
        \item \texttt{cp \textit{source} \textit{destination}} für copy: kopiert eine Datei
        \item \texttt{cp -r \textit{source} \textit{destination}}: kopiert ein Verzeichnis rekursiv
        \item das Ziel kann ein Verzeichnis oder der exakte Pfad sein
        \item \texttt{\textit{source}} können mehrere Dateien sein, der letzte Pfad zählt als \texttt{\textit{destination}}
        \item \texttt{mv \textit{source} \textit{destination}} für move: verschiebt order benennt eine Datei um
        \item Ziel kann wir bei \texttt{cp} sein
        \item \texttt{rm \textit{file}} für remove: löscht eine Datei
        \item \texttt{rm -r \textit{file}}: löscht ein Verzeichnis rekursiv
        \item \texttt{rmdir \textit{directory}} für remove directory: löscht ein leeres Verzeichnis
        \item \texttt{rm -r} kann statt \texttt{rmdir} verwendet werden
      \end{itemize}
    \end{frame}

    \begin{frame}
      Beispiel:\\
      \texttt{\$ ls\\
              a\\
              \$ cp a b\\
              \$ ls\\
              a  b\\
              \$ mv b c\\
              \$ ls\\
              a  c\\
              \$ rm a\\
              \$ ls\\
              c}
    \end{frame}

    \begin{frame}{\texttt{cat}, \texttt{less}, \texttt{grep}, \texttt{echo}}
      \begin{itemize}
        \item \texttt{cat [\textit{file}]} für concatenate: gibt den Inhalt einer (oder mehrerer) Dateien aus
        \item \texttt{less [\textit{file}]} (besser als \texttt{more}): zeigt eine Datei in einer navigablen Form an
        \item \texttt{grep \textit{pattern} [\textit{file}]} für ???: sucht nach einem Muster
        \item \texttt{grep -i \textit{pattern} [\textit{file}]}: ignoriert Groß- und Kleinschreibung
        \item \texttt{grep -r \textit{pattern} \textit{directory}}: sucht rekursiv in allen Dateien
        \item \texttt{echo \textit{text}}: gibt einen Text aus
      \end{itemize}
    \end{frame}

  \subsection{Nützliche Shell Features}
    \begin{frame}{Allgemeines}
      \begin{itemize}
        \item pipes\\
        \item ctrl-c\\
        \item ctrl-d\\
        \item .., .\\
        \item >, >>, <\\
        \item glob (*, ...)
      \end{itemize}
    \end{frame}

\section{git}
\begin{frame}{git}
  \tableofcontents[sectionstyle=show/hide,
                   subsectionstyle=show/show/hide,
                   subsubsectionstyle=show/show/show]
\end{frame}

\subsection{Warum Versionskontrolle?}
\begin{frame}{Warum Versionskontrolle?}
\end{frame}

\subsection{Hoster}
\begin{frame}{Hoster}
\end{frame}

\subsection{Befehle}
\begin{frame}{Repository erstellen}
  \begin{itemize}
    \item \texttt{git init}  Erzeugt ein leeres Repository im jetzigen Ordner.
    \item \texttt{git clone [\textit{url}]} Kopiert ein Repository aus dem Internet
  \end{itemize}
\end{frame}

\begin{frame}{Informationen abrufen}
  \begin{itemize}
    \item \texttt{git status} Zeigt an, welche Dateien geändert wurden und welche bereits im Index sind
    \item \texttt{git log}    Zeigt alle gespeicherten Commits an
  \end{itemize}
\end{frame}

\begin{frame}{Dateien zum Index hinzufügen}
  \begin{itemize}
    \item \texttt{git add} Fügt eine Datei oder einen Ordner (mit Inhalt) zum Index hinzu
    \item \texttt{git rm}  Löscht eine Datei aus dem Ordner und schreibt die Löschung in den Index
    \item \texttt{git mv}  Genauso, aber die Datei wird verschoben
  \end{itemize}
\end{frame}

\begin{frame}{Commits erstellen}
  \begin{itemize}
    \item \texttt{git commit} Speichert die Änderungen im Index als Commit ab
  \end{itemize}
\end{frame}

\begin{frame}{Änderungen runter-/hochladen}
  \begin{itemize}
    \item \texttt{git pull} Neue Commits runterladen\\
                            Falls man noch neue lokale Commits hat führt git einen \textit{merge} durch
    \item \texttt{git push} Neue Commits hochladen\\
                            Geht nur, wenn keine neuen Commits im zentralen Repository sind\\
                            Wenn ja, erst einmal \texttt{git pull} verwenden
  \end{itemize}
\end{frame}

\begin{frame}{Manuell mergen}
  \begin{itemize}
    \item \texttt{git mergetool} Startet ein Programm, mit dem man manuell mergen kann, falls die Automatik nicht funktioniert
  \end{itemize}
\end{frame}

  \section{Python}
    \subsection{Sprache}
      \begin{frame}{ipython}
      \end{frame}
      
      \begin{frame}{variablen, operatoren}
      \end{frame}
      
      \begin{frame}{(), [], \{\}}
      \end{frame}
      
      \begin{frame}{if, elif, else}
      \end{frame}
      
      \begin{frame}{for, while}
      \end{frame}
      
      \begin{frame}{def, Funktionsaufrufe}
      \end{frame}
      
      \begin{frame}{import, from, as}
      \end{frame}
    
    \subsection{Bibliotheken}
      \begin{frame}{numpy}
      \end{frame}
      
      \begin{frame}{scipy}
      \end{frame}
      
      \begin{frame}{matplotlib}
      \end{frame}
      
      \begin{frame}{uncertainties}
      \end{frame}
      
      \begin{frame}{pylab}
      \end{frame}

\chapter{LaTeX}
\end{document}
