\chapter{Git}
\begin{center}
    \includegraphics[width=120px]{img/git.pdf} \\
    \textbf{\href{http://git-scm.com}{www.git-scm.com}}
\end{center}

\section{Warum Versionskontrolle?}

Wissenschaftliche Projekte werden meist in Kollaboration mit anderen durchgeführt.
Die einzelnen Wissenschaftler arbeiten dabei oft an völlig verschiedenen Orten.
Es müssen längere Dokumente verfasst werden, oft werden Programme entwickelt.
Arbeit und Ergebnisse sollten klar protokolliert und reproduzierbar sein.
Auch sollten stets Backups des Projekts existieren.

Ein Versionskontroll-System ist ein Programm, das genau diese Aufgaben erfüllt.
Das Konzept der Versionskontrolle kam ursprünglich in der Software-Entwicklung auf.
Hier besteht wegen großer Teams und vielen Code-Zeilen ein besonderer Bedarf.

Diese Programme können aber auch außerhalb der Software-Entwicklung
effektiv eingesetzt werden.

Bekannte Versionskontroll-Systeme sind z.B.:
\begin{itemize}
  \item CVS
  \item SVN
  \item Mercurial
  \item \textbf{Git}
\end{itemize}

Besonders \texttt{git} hat sich in den letzten Jahren durchgesetzt.
Es weist einige nützliche Vorteile auf:
\begin{itemize}
  \item Es ist sehr schnell
  \item Jeder Nutzer besitzt lokal die gesamte Vergangenheit des Projekts.
    So kann auch ohne Internetverbindung effektiv gearbeitet werden.
  \item Es existieren gute Online-Services, die \texttt{git} unterstützten.
\end{itemize}

% Warum Versionskontrolle?
%
% - Protokollierung
% - Kollaboration
% - Backup
%
% Warum Git?
%
% - Funktioniert auch ohne zentrales Repository
% - Sehr schnell
% - Hat sich durchgesetzt

\section{Hosting}

\begin{itemize}
  \item $
    \begin{array}{l}
      \includegraphics[height=18px]{img/octocat.jpg}
    \end{array} $ \textbf{Github}

  \item $
    \begin{array}{l}
      \includegraphics[height=18px]{img/bitbucket.png}
    \end{array} $ \textbf{Bitbucket}

  \item \textbf{eigenes Repository}
\end{itemize}

\section{Befehle}

% TODO expand
\begin{itemize}
  \item \texttt{git init}
  \item \texttt{git clone}
  \item \texttt{git status}
  \item \texttt{git add}
  \item \texttt{git log}
  \item \texttt{git mv}
  \item \texttt{git rm}
  \item \texttt{git commit}
  \item \texttt{git push}
  \item \texttt{git pull}
  \item \texttt{git mergetool}
\end{itemize}

\section{Best Practices}

Einige Punkte, die man bei der Benutzung von \texttt{git} beachten sollte:
\begin{itemize}
  \item Automatisch generierte, sich oft ändernde Dateien sollten
    möglichst nicht im Repository gespeichert werden
    $\Rightarrow$ bläht sonst die Größe auf
  \item Git funktioniert am Besten mit reinen Text-Dateien. (\LaTeX-Code, Programme, etc.)
    Binäre Formate wie Word- oder Excel-Dateien können zwar auch abgespeichert werden,
    die Änderungen können aber nicht bequem eingesehen werden und die Größe des
    Repositories steigt unter Umständen an.
\end{itemize}

\section{Weiterführende Links}
\begin{itemize}
  \item \texttt{man git}, \texttt{man gittutorial}
  \item Pro Git \url{http://git-scm.com/book}
  \item Git - Documentation \url{http://git-scm.com/doc}
  \item Git Immersion \url{http://http://gitimmersion.com/}
  \item Easy Version Control with Git
    \url{http://net.tutsplus.com/tutorials/other/easy-version-control-with-git/}
\end{itemize}
