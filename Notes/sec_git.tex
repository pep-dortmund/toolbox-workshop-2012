\chapter{Git}
\begin{center}
    \includegraphics[width=120px]{img/git.pdf} \\
    \textbf{\href{http://git-scm.com}{www.git-scm.com}}
\end{center}

\section{Warum Versionskontrolle?}
Wissenschaftliche Projekte werden meist in Kollaboration durchgeführt.
Die einzelnen Wissenschaftler arbeiten dabei oft an völlig verschiedenen Orten.
Es müssen längere Dokumente verfasst werden; oft werden Programme entwickelt.
Arbeit und Ergebnisse sollten klar protokolliert und reproduzierbar sein.
Auch sollten stets Backups des Projekts existieren.

Ein Versionskontroll-System ist ein Programm, das genau diese Aufgaben erfüllt.
Das Konzept der Versionskontrolle kam ursprünglich in der Software-Entwicklung auf.
Hier besteht wegen großer Teams und vieler Code-Zeilen ein besonderer Bedarf.

Diese Programme können aber auch außerhalb der Software-Entwicklung effektiv eingesetzt werden.

Bekannte Versionskontroll-Systeme sind z.B.:
\begin{itemize}
  \item CVS
  \item SVN
  \item Mercurial (\texttt{hg})
  \item \textbf{\texttt{git}}
\end{itemize}

Besonders \texttt{git} setzt sich in den letzten Jahren immer mehr durch.
Es weist einige nützliche Vorteile auf:
\begin{itemize}
  \item Es ist sehr schnell
  \item Jeder Nutzer besitzt lokal die gesamte Vergangenheit des Projekts. (\textit{Distributed Version Control System})
    So kann auch ohne Internetverbindung effektiv gearbeitet werden.
  \item Das automatische Zusammenfügen mehrerer Änderungen (\textit{mergen}) ist sehr gut gelöst.
  \item Es existieren gute Online-Services, die \texttt{git} unterstützen.
\end{itemize}

% Warum Versionskontrolle?
%
% - Protokollierung
% - Kollaboration
% - Backup
%
% Warum Git?
%
% - Funktioniert auch ohne zentrales Repository
% - Sehr schnell
% - Hat sich durchgesetzt

\section{Überblick}

% TODO Needs some pictures

Git ist vorrangig ein Kommandozeilenprogramm.
Es ist aber kein Problem wenn man die Kommandozeile lieber meidet: Es existieren zahlreiche grafische Oberflächen, die einem die tägliche Arbeit erleichtern können.
Trotzdem sollte man zuerst die Kommandozeilen-Befehle lernen, da man so am schnellsten ein Gefühl für die Funktionsweise von \texttt{git} bekommt.
Außerdem nutzen Beispiele und Erklärungen im Internet fast ausschließlich die Kommandozeilen-Syntax.

Einige Begriffe sollten man kennen, bevor man sich mit den Befehlen vertraut macht:

Ein \texttt{git}-Repository existiert innerhalb eines Verzeichnisses (dem \textit{Working directory}) im Dateisystem.
Bis auf den Ordner \verb|.git| innerhalb dieses Verzeichnisses merkt man erst einmal nichts von \texttt{git}.

Hat man einige Änderungen an seinem Projekt vorgenommen und möchte diese speichern oder mit anderen teilen, so fügt man sie mit einem Befehl zum sogenannten \textit{Index} hinzu.
Ist man mit der Sammlung zufrieden, kann man sie als \textit{Commit} abspeichern.
Dabei gibt man in der Regel eine kurze Erklärung, was am Projekt geändert wurde.
(Mehrere Commits nacheinander bilden eine Kette)

Man kann seine Commits anschließend hochladen.
Interessant wird es, wenn man die Commits anderer herunterlädt.
Git fügt diese dann automatisch mit den eigenen zusammen.

\section{Hosting}

Es lohnt sich, ein Hauptrepository im Internet zu haben, auf das alle zugreifen können.
Die bekanntesten \texttt{git}-Hoster sind 
\begin{itemize}
  \item $
    \begin{array}{l}
      \includegraphics[height=18px]{img/octocat.jpg}
    \end{array} $ \textbf{Github}

  \item $
    \begin{array}{l}
      \includegraphics[height=18px]{img/bitbucket.png}
    \end{array} $ \textbf{Bitbucket}
\end{itemize}
Github bietet kostenlose Repositories an, wenn man den Inhalt öffentlich macht.
Bitbucket bietet zusätzlich auch kostenlose private Repositories an, die man nur mit einem Passwort einsehen kann.
Wer einen eigenen Server besitzt kann \texttt{git} auch darauf einrichten.
Hier empfiehlt es sich, \href{https://github.com/sitaramc/gitolite}{\texttt{gitolite}} zu benutzen.

\section{Befehle}

\subsection{\texttt{git init}}
Mit diesem Befehl wird aus dem jetzigen Verzeichnis ein leeres Repository.

\subsection{\texttt{git clone}}
Kopiert ein Repository (meist aus dem Internet).
Als Parameter muss die URL des Repositories übergeben werden.
Beispiel:
\begin{verbatim}
git clone http://github.com/user/repo
\end{verbatim}

\subsection{\texttt{git status}}
Zeigt an, welche Dateien seit dem letzten Commit verändert wurden, welche neu hinzugekommen sind und welche bereits für den nächsten Commit vorbereitet wurden.

\subsection{\texttt{git add}}
Fügt eine Datei oder einen Ordner zum \textit{Index} hinzu. Das bedeutet, dass die Dateien beim nächsten Commit abgespeichert werden.
Bei Ordnern werden auch alle Unterordner und Dateien darin hinzugefügt.
Wenn man also
\begin{verbatim}
git add .
\end{verbatim}
im obersten Verzeichnis des Repositories aufruft, werden alle Dateien hinzugefügt.

\subsection{\texttt{git log}}
Zeigt alle Commits im Repository an.
Man kann die Nachricht sehen, mit der sie gespeichert wurden, sowie wann und von wem sie erstellt wurden.
Es gibt zahlreiche Optionen, mit denen man die Ausgabe anpassen kann.
(siehe \verb|git help log|)

\subsection{\texttt{git rm}, \texttt{git mv}}
Wenn man Dateien löscht oder sie bewegt muss man git mitteilen, dass man die Datei (an der alten Stelle) tatsächlich löschen will.
Das geht am Besten in einem Schritt, wenn man sofort \texttt{git rm} und \texttt{git mv} benutzt.

\subsection{\texttt{git commit}}
Speichert die Änderungen im Index als Commit ab.
Öffnet auch einen Editor, damit man beschreiben kann, was man geändert hat.
Meistens reicht es aber auch die Änderung mit dem Parameter \verb|-m| direkt einzugeben:
\begin{verbatim}
git commit -m "Ein paar Änderungen"
\end{verbatim}
Das geht schneller.

\subsection{\texttt{git push}}
Schickt die neuen Commits zu dem Repository, das man zu Anfang mit \verb|git clone| kopiert hat.
Hat man das Repository ursprünglich mit \verb|git init| erstellt, so muss man erstmal ein Zielrepository einstellen:
\begin{verbatim}
git remote add --track master origin [URL]
git push origin master
\end{verbatim}
Danach ist das Repository abgespeichert und man kann immer \verb|git push| benutzen.

Wenn bereits ein Anderer neue Änderungen ins Repository hochgeladen hat, kommt eine Fehlermeldung.
Nun muss man erst einmal \verb|git pull| benutzen (nächster Befehl) und sicherstellen, dass die Änderungen vernünftig zusammengefügt werden.
Danach kann man das Ganze mit \verb|git push| hochladen.

\subsection{\texttt{git pull}}
Holt neue Änderungen aus dem Repository.
Man kann sich dann mit \verb|git log| ansehen, was gemacht wurde.
Hat man im eigenen Repository noch neue Änderungen, so wird \texttt{git} versuchen, diese automatisch mit den neuen zu mergen.

\subsection{\texttt{git mergetool}}
Wurde dabei dieselbe Zeile von mehreren Leuten geändert, so kommt es zu einem \textit{merge conflict}.
Hierbei sollte man \verb|git mergetool| eingeben, woraufhin sich ein Programm öffnet mit dessen Hilfe man die kritischen Stellen finden und das Problem manuell beseitigen kann.
Wer die Installationsanleitung befolgt hat, hat das Programm \verb|kdiff3| installiert, was genau dafür geschrieben wurde.
Hinterher muss man noch \verb|git commit| eingeben, um den merge abzuschließen.

\subsection{\texttt{.gitignore}}
Wen es stört, dass Dateien, die man gar nicht im Repository haben will, ständig bei \verb|git status| angezeigt werden, kann die Datei \verb|.gitignore| im obersten Verzeichnis des Repositories erstellen.
Darin kann man Regeln definieren, nach denen Dateien von \verb|git| ignoriert werden sollen:
\begin{verbatim}
*.pdf
build/*
\end{verbatim}
In diesem Beispiel werden alle Dateien ignoriert, die auf \verb|.pdf| enden, oder die in einem Ordner mit dem Namen \verb|build| liegen.

\section{Best Practices}

Einige Punkte, die man bei der Benutzung von \texttt{git} beachten sollte:
\begin{itemize}
  \item Automatisch generierte, sich oft ändernde Dateien sollten möglichst nicht im Repository gespeichert werden
    $\Rightarrow$ bläht sonst die Größe auf
  \item Git funktioniert am Besten mit reinen Text-Dateien (\LaTeX-Code, Programme, etc.).
    Binäre Formate wie Word- oder Excel-Dateien können zwar auch abgespeichert werden, das automatische Mergen funktioniert aber kaum, die Änderungen können nicht bequem eingesehen werden und die Größe des Repositories steigt unter Umständen an.
\end{itemize}

\section{Weiterführende Links}
\begin{itemize}
  \item \texttt{man git}, \texttt{man gittutorial}
  \item Pro Git: \url{http://git-scm.com/book}
  \item Git - Documentation: \url{http://git-scm.com/doc}
  \item Git Immersion: \url{http://gitimmersion.com/}
  \item Easy Version Control with Git: \url{http://net.tutsplus.com/tutorials/other/easy-version-control-with-git/}
  \item Git (Wikipedia): \url{http://en.wikipedia.org/wiki/Git_(software)}
\end{itemize}
